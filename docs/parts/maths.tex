\section{Math: Basic intuitions}
Matrices are a rectangles of elements. Hence they have width and height (dimensions).

They come up as a handy notation for sets of equations (no need if there is just one equation). 

\subsection{Examples}
\textbf{One equation, one Variable}
We are told that the relation mass of flour (kilograms) to volume (in litres) of water for a particular recipe is: $$f = 2\,w + 3$$
The change or derivative is $$\frac{df}{dw} = 2x$$

\textbf{One equation, many variables}
Let's create an example where the weight of flour is always double the volume of water, $\frac{1}{5}$ of sunflower oil, and $0.01$ potasium carbonate. 

$$f(w,so,k) = 2\,w + \frac{1}{5}\,so + 0.01\,k$$

Now $f$ depends on many ingredients, instead of $f(w)$ we have $f(w,so,k,\ldots)$. In standard notation this is $f(x,y,z,\ldots)$. 

The derivative has to contemplate each variable,

\begin{align*}
  \nabla f(w,so,k) &= \frac{df}{dw} + \frac{df}{dso} \frac{df}{dk}\\
&=2 + \frac{1}{5} + 0.01
\end{align*}
Now $df$ is normally written $\nabla f$ but it means the same, derivatives respect to each component. Adding up we get a total change $df$, but a different view is keeping them separate, and hence the information on each direction (axis) isn't loose.

\begin{align*}
\nabla f(w,so,k) =
\begin{bmatrix}
   \frac{df}{dw} \\ \frac{df}{dso} \\ \frac{df}{dk}
\end{bmatrix}
=
\begin{bmatrix}
  2 \\ \frac{1}{5} \\ 0.01
\end{bmatrix}
\end{align*}

This is part of \textit{multivariate calculus} needed for DL.

\begin{center}
\begin{tabular}{cc}
  $df(x)$ & $\nabla f(x,y,z,\ldots)$\\
  $\frac{df}{dx}$ & $\frac{\partial f}{\partial x}+\frac{\partial f}{\partial z}+\ldots$
\end{tabular}
\end{center}

It was shown how many variables naturally lead to vectors - this is, they can be arranged as vectors.

\textbf{Many equations, one variable}

What happens with many equations? 

A thousand recipes containing Flour and Water in different proportions is needed for a dinner. There are $1000$ equations similar to the one above $f= k\,w + b$. $k$, $c$ being constants. $w$ could very well be a single pixel of an image. 

This can be represented as a matrix:
\begin{align*}
\begin{bmatrix}
f_1\\
f_2\\
\vdots\\
  f_{1000}
\end{bmatrix}
=
\begin{bmatrix}
k_1 & 0 & \hdots & 0_n \\
0 & k_2 & \hdots & 0_n \\
\vdots & & \hdots & \vdots \\
  0 & 0 & 0 & k_{1000}
\end{bmatrix}
\begin{bmatrix}
W_1\\
W_2\\
\vdots\\
  W_{1000}
\end{bmatrix}
+
\begin{bmatrix}
b_1\\
b_2\\
\vdots\\
  b_{1000}
\end{bmatrix}
\end{align*}

What would mean to find $d\mathbf{F}$? The equation $i^{th}$ is $f_i(w_i) = w_i\,k_i + b_i$, then $df_i = \frac{df_i}{dw_i} = k_i$ (and terms different from $i$ are $0$). $$d\mathbf{F} = \mathbf{k}$$.

What would happen if each $f_i$ is in a exponent, for example? $f_i(w_i) =  e^{w_i\,k_i + b_i}$. How are the $1000$ equations represented? Just $e^\mathbf{F}$ would indicate that. What is the derivative?

\begin{equation*}\frac{df}{dw} = e^{w_i\,k_i + b_i}\, k_i\end{equation*}. Then $$\frac{d\mathbf{F}}{dW}= \mathbf{k} \, e^\mathbf{F}$$

  In this way many derivatives can be found. For example $d\mathbf{x}^T\mathbf{x} = 2 \mathbf{x}$. Because $\mathbf{x}^T\mathbf{x}$ is a representation of a set of equations of this kind: $x^2$. A \href{https://www.gatsby.ucl.ac.uk/teaching/courses/sntn/sntn-2017/resources/Matrix_derivatives_cribsheet.pdf}{list of useful derivatives}.

For many variables, the matrix $\mathbf{k}$ has numbers other than $0$.
